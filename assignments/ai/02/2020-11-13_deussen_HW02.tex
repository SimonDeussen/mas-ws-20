\documentclass{article}

\usepackage{float}
\usepackage{graphicx}
\usepackage{listings}% http://ctan.org/pkg/listings
\lstset{
  basicstyle=\ttfamily,
  mathescape
}

\title{MAS AI Assignment 2}
\author{Simon Deussen}

\begin{document}
\pagenumbering{gobble}
\maketitle
\pagenumbering{arabic}

\section*{Task 1}
In the first part of my answer I will explain why measuring the runtime in seconds does not work,
  and in the second part I am going to explain the notation.

We want to compare the runtime of algorithms, for this it is important that the classification 
works \emph{regardless of the computer running it}. This requirement immediately dismisses the measuring
in seconds. How would it be possible to map the runtime in seconds between different computer with different
instruction sets and clock speeds? 

Secondly, an algorithm works on a given input. If we only look at the runtime, we can not possibly know how any 
algorithm works with different amounts of input data. Because of those reasons,
we can see that working with absolute runtime in seconds is only getting complicated and wont allow us important
comparisons. 
\\

The special \emph{Big o notation} is a measure that describes how many steps an algorithm has to perform
in relation to the amount of input data. By looking at abstract steps, this notation removes many nasty 
details from the discussion of algorithmic runtime. It works in a way that it describes to growth
of needed steps in relation to number of input data. Which allows an easy comparisons of algorithms.

\section*{Task 2}

Prove the following:
\begin{itemize}
  \item 
  $f(n) = 100n^2 \in O(n^2)$ 
  
  is True because: 
  
  $f(n) \le  c g(n^2) \forall c > 100$
  \item 
  $f(n) = n^6 + 100n^5 \in O(n^6) $ 
  
  is True because: 
  
  $f(n) \le  c g(n^6) \forall c > 100$
  
  because

  $cn^6 > 100n^5 \forall c > 100$
\end{itemize}


\section*{Task 3}

First, lets annotate the given code with the cost per line and execution times:
\begin{lstlisting}
code                            cost    times
sum = 0                         $c_0$        $1$
for i in range(0, J):           $c_1$        $J$
  for j in range(0, K):         $c_2$        $J*K$ 
    if arr[i][j] <= ANY_CONST:  $c_3$        $n_0 \le J*K$ 
      sum = sim + arr[i][j]     $c_4$        $n_1 \le J*K$

print(sum)                      $c_5$        $1$
\end{lstlisting}

We see that the total cost equals: $c_0 + c_1 J + c_2 JK + c_3 n_0 + c_4 n_1 + c_5 \forall n_0, n_1 \le JK$
For the \emph{Big O notation} only the biggest term is relevant: $c_2 JK$
Further, the constant is irrelevant and we assume the worst case, $J=K$ which leaves: $J^2$.
Because there is a $g(n) = cn^2 > J^2 \forall c > 0$ we can say, this code has the complexity of $O(n^2)$.


\section*{Task 4}
For this task we have to calculate all the entries of given table. Where i could invert the given function

\begin{table}[H]
  \begin{tabular}{llllllll}
                       & 1 second & 1 minute & 1 hour & 1 day & 1 month & 1 year & 1 century \\
  $lg(n)$               & $10^{60}$&$10^{120}$&$10^{180}$&$10^{204}$&$10^{234}$&$10^{246}$&$10^{256}$\\
  $\sqrt{n}$           &$10^{12}$ & $3.6*10^{15}$&$1.4*10^{19}$&$7.5*10^{21}$&$6.7*10^{24}$&$9.7*10^{26}$&$9.7*10^{28}$\\
  $n$                  & $10^{6}$&$6*10^{7}$&$3.6*10^{9}$&$8.6* 10^{10}$&$2.6* 10^{12}$&$3.1* 10^{13}$&$3.1* 10^{14}$\\
  $n lg(n)$            & & &        &       &         &        &           \\
  $n^2$                &$10^{3}$  &$7.7*10^{3}$&$6.0*10^{4}$&$2.9*10^{5}$&$1.6*10^{6}$&$5.6*10^{5}$&$1.7*10^{7}$\\
  $n^3$                &$10^{2}$  &$3.9*10^{2}$&$1.5*10^{3}$&$4.4*10^{3}$&$1.3*10^{4}$&$3.14*10^{4}$&$6.8*10^{4}$\\
  $2^n$                &  19.9    & 25.8          & 31.7        &    36.3   & 41.2        & 44.8       & 48.1          \\
  $n!$                 &    9     &     10     &     12   &    13   &      15   &    16    &      16    
  \end{tabular}
  \end{table}

  \begin{description}
    \item[$n lg(n)$] I do not know how to calculate this efficiently.
    \item[$2^n$] The inverse is: $log_{2}(x)$
    \item[$n!$] I made an approximation (brute force) using a python script. 
  \end{description}


% \bibliography{stuff} 
% \bibliographystyle{ieeetr}

\end{document}