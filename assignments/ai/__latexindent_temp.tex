\documentclass{article}

\title{MAS AI Assignment 1}
\date{}
\author{Simon Deussen}

\begin{document}
  \pagenumbering{gobble}
  \maketitle
  \pagenumbering{arabic}

\section*{Task 1}
here i will make a very nice mindmap


\section*{Task 2}
What are  \emph{reflex-, model-, goal-} and \emph{utility-based} agents? Answers based on \emph{Artificial intelligence: a modern approach} \cite{russell_norvig_2014} 

\begin{description}
    \item[reflex-based agent] 
    The simplest kind of agent. \
    A reflex-based agents looks up a action based only on current percepted input. 
    The behaviour of this agent can be written as \textbf{condition-action} rules.
    
    \item[model-based agent] 
    This agent has the ability to create a \textbf{world model} for keeping track of external state.
    The agents perception is used to update the world model. Besides the state of the world, 
    the agent also model what each action can do. Based on this world state and the current 
    state it chooses the next action in the same way as the reflex-based agent.

    \item[goal-based agent] This agent models the external state the same way as the model-based \
    agent but the big difference is the introduction of goals. Those goals are used to
    choose the next action instead of the condition-action rules. For achiving its goals,
    this agent needs to \textbf{plan} ahead and \textbf{search} for the best action sequence.
 
    \item[utility-based agent]
     A goal based agents can only decide in a binary if the goal is fullfisurement lled. 
     This utility-based agent has an inbuild \textbf{performance measurement} estimating how good the next 
     actions in the context of fullfilling its goal are.
  \end{description}


\section*{Task 3}
\emph{"Braitenberg vehicles are simple autonomous agents that move around based on sensor input."}\cite{harmendeweerd:1} With clever combinations
 Braitenberg was able to create simple behaviour patterns like light seeking or circling around objects. 
 By just attaching a sensor to an actuator. Check this website for a cool simulation\cite{harmendeweerd:1}. 
 Because no computation happens and each action is directly correlated to a specific sensorical input, 
 I would classify those agents as \textbf{reflex-based}.

\section*{Task 4}
What are  \emph{reflex-, model-, goal-} and \emph{utility-based} agents?

\bibliography{stuff} 
\bibliographystyle{ieeetr}

\end{document}