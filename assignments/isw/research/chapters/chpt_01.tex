%!TEX root = ../annotations.tex
\definecolor{lightgray}{gray}{0.96}
\newcommand*\rot{\rotatebox{90}}
\begin{document}
    \chapter{Introduction to agricultural robots}

    This report focuses on the problems in the development and deployment of agricultural robots and tries evaluate if system of multiple robots are able to bring advantages into this area.

    The global population is growing, at the same time the amount of arable land remains the same. Until 2050, the global population will surpass 9.1 billion people - which are going to mostly live in cities and are dependant on farmers fulfilling their steady demands to high quality foods. Even today are millions of people malnourished. With more people dependant on the limited food sources, this situation will get worse \cite{FAO2021}
    
    Following the needs of more people, it will be important to grow even more food than currently obtainable. Because the existing land is limited, it is important to increase the efficiency and output of farming. 

    Not just the higher demand pressures farmers, but also another problem: Farming is highly dependant on human labour. Working in fields is demanding on the body and often detrimental for the health of the workers. Because of this, the available workforce shrinks by following other sectors of work instead. This makes quality work harder and harder to find for the farmers. \cite{Vasconez2019}

    Big scale conventional farming can of course produce high amounts of foods, but not without a hefty environmental footprint. Usage of pesticides and herbicides - often distributed blind over whole fields - is not sustainable and should be reduced if possible. \cite{ConesaMunoz2015}

    Those three points make already a convincing argument for automating the process of farming by robots - but why are our fields and greenhouses still taken care of by humans? 

    In contrast to modern industry, were factories work with minimal human labour and completely automated assembly lines, it seems that in agriculture some problem arise which still hinders the use of autonomous robots. 

    

    % Even though there are already fully automated factories and assembly lines, automations and robotics lack behind in the areas of agriculture. 
    % Because the there are no systems which are producing adequate results. 

    % The environment in which robots would have to work on fields are highly dynamic, unstructured, dirty, dust, highly stochastic \cite{Bechar2016}

    % There are some system which are able to do some task in open field tests, but even here is the performance too bad \cite{Bac2014}

    % In agriculture are many different tasks which need to be done, planting, scouting, plant care such as pruning, weed removal, and finally harvesting. 
    
    % Harvesting, what all is needed, sensing, processing the data, planning how to access the fruit, moving to the fruit and detaching it.  putting the fruit somewhere. This works okay with current technology as long as no leaves come between the cameras and the fruit. 
   
    % Most popular research topic is harvesting - even a robot fully capable of harvesting would not be used most of the year, because of the seasonality of the growth cycles.

    % All in all many problems hindering the adaption of automation in agriculture. 

    % It is Nevertheless an exiting field like all of field robots because of the need for automation here and the future potential in it. \cite{Thorpe2001}



    \chapter{Description of agricultural robots and multi-robot systems.}

    \section{What are the task of agricultural robots and what makes the usage of robots in agriculture so hard?}


    The types of robots needed for agriculture are real generalists, many different tasks occur in farming. This section will give the reader a quick overview of the challenges the robots will meet.

    Robots in agriculture have multiple tasks. They are all about plant care and finally harvesting the produce. Following tasks are typical for agricultural robots: 
    \begin{description}
        \item[Scouting] Scouting is the process of monitoring the plant health, the ripeness, the estimated yield, different climate parameters like temperature and humidity and also weeds or other diseases.
        \item[Pruning] Pruning is a type of plant care were for examples leaves get removed for a more effective grow process.
        \item[Weeding] The thermal, mechanical or chemical process of removing weeds.
        \item[Application of chemicals] The delivery of different agrochemicals like fertilizers, herbicides and pesticides to plants or parts of the field.
        \item[Harvesting] Harvesting is the final stage in the growth cycle. Now the ripe fruit has to be detached and collected. Harvesting is no problem for bulk produce like wheat or corn, but for high value crops like sweet pepper or tomatoes, is harvesting the hardest part to automate.
    \end{description}

    To achieve any of those task, the robots also have to be able to do many supporting task like self-localization, navigation, path and task planning and ensuring safety during operation. Important are also the systems for mobility, including steering, obstacle avoidance and control

    Agricultural have to work in a variety of different environments, the most notable are open field, greenhouses, orchards, vineyards and forests.

    Before coming to the problem of agricultural robots, one should have a look about the advantages agricultural robots will bring if deployed properly. Utilizing robots will be the most cost effective method for all task to be done. Human labor is the biggest factor in crop costs and along side saving in the use of agrochemicals shows big potential for cost savings. At the same time the use of robots will improve the quality of the growing crops as well as uniformity. Not just this, but the robots will increase each production and profits. Having an automated workforce, which needs to capture all kinds of data for working properly has another benefit: The aggregated data will help the farmers make smarter, faster and better decisions. In comparison to conventional methods, robots will reduce environmental strain and thus increase the sustainability of farming. \cite{Bechar2016}

    In the mind map \hyperlink{fig:mindmap_agribots}{\emph{Agricultural robots}} I showcase the most important points for more convenient access and reminding.

    \vspace{10px}
    \hrule
    \vspace{10px}

    \begin{tikzpicture}[grow cyclic, text width=2.7cm, align=flush center,
        level 1/.style={level distance=4.2cm,sibling angle=90},
        level 2/.style={level distance=2.2cm,sibling angle=50}]
                child{node {Fertilizers}}
    \node{\hypertarget{fig:mindmap_agribots}{\textbf{\huge Agricultural robots}}}
        child { node {\textbf{Primary tasks}}
            child { node {\footnotesize Scouting}}
            child { node {\footnotesize Pruning}}
            child { node {\footnotesize Weeding}}
            child { node {\footnotesize Harvesting}
                child { node {\footnotesize Sensing and detection in 3D}}
                child { node {\footnotesize Action planning}}
                child { node {\footnotesize Manipulation}}
            }
            child { node {\footnotesize Application of agrochemicals}}
        }
        child { node {\textbf{Secondary tasks}}
            child { node {\footnotesize Self-localization \& navigation}}
            child { node {\footnotesize Mobility}
                child { node {\footnotesize Steering}}
                child { node {\footnotesize Obstacle avoidance}}
                child { node {\footnotesize Control}}
            }
            child { node {\footnotesize Path planning}}
            child { node {\footnotesize Logistics}}
            child { node {\footnotesize Safety}}
        }
        child { node {\textbf{Environment types}}
            child { node {\footnotesize Open fields}}
            child { node {\footnotesize Greenhouses}}
            child { node {\footnotesize Orchards}}
            child { node {\footnotesize Vineyards}}
            child { node {\footnotesize Forests}}
        }
        child { node {\textbf{Motivation}}
            child { node {\footnotesize  Higher efficiency and higher yield}}
            child { node {\footnotesize Cost reduction}}
            child { node {\footnotesize Lessen usage of agrochemicals}}
            child { node {\footnotesize Higher quality}}
            child { node {\footnotesize Compensate lack of workforce}}
        }
        ;
    
    \end{tikzpicture}
    \vspace{10px}
    \hrule
    \vspace{10px}

    Robots working in the agricultural domain have to master a multitude of difficult tasks in a ever changing environment. \cite{Bechar2016} have compared the complexity of the domain of agriculture robots to other robotics domains by creating four groups along the axis of structured and unstructured objects as well as structured and unstructured environments: 
    \begin{table}[h]
        \captionof{table}{Classification of objects and environments} \label{table:objects_and_tasks} 
        \begin{tabular}{@{}lll@{}}
        \toprule
                             & \textbf{Structured environment} & \textbf{Unstructured environment}                      \\ \midrule
        \textbf{Structured objects}   & Industrial domain      & Military, underwater \& mining domain \\
        \textbf{Unstructured objects} & Medical domain         & Agricultural domain                           \\ \bottomrule
        \end{tabular}
        \end{table}

    Compared to other robotics domains, robots in agriculture have to combat an unstructured environment \textbf{and} unstructured objects. Every plant and every fruit is different from each other, every meter of ground is complex and rough to navigate - every process is highly stochastic. Plants grow wild and the exposed environment will worsen the conditions additionally. For reliable and robust robots should be able to work in all kinds of weather conditions, from dusty, glaring sunshine to rain - and even in the night. Compared to a robot working in a highly engineering assembly in perfect lighting and climate control, we can see that the problems here are orders of magnitude harder to solve. 
    Because of this, robots in agriculture are so far behind their counterparts in the industrial domain. There are many existing robot projects like \cite{Arad2020, Chamanbaz2017, ConesaMunoz2015, Herck2020, Lehnert2020, Wu2020} which show successful harvesting of a couple of crops. But none of those robot system can to more than one specific task. They can navigate through the fields and harvest with varying performance, but struggle to reach economic viability - even after decades of intensive review. For a detailed review of actual performance, refer to \cite{Bac2014, Bechar2017}.

        \section{Agricultural robots in the bigger field of robotic research.}

        \vspace{10px}
        \hrule
        \vspace{10px}

        \begin{tikzpicture}[text width=8cm, align=flush center, level distance=1cm, sibling distance=4cm]
            \node[label=above:\textbf{Categorization of agricultural robots}]{\hypertarget{fig:category_mindmap}{Types of robots}}
                child { node {Industrial robots}}
                child { node {Service robots}
                    child {node {Personal service robots}}
                    child {node {Professional service robots}
                        child {node {Field robots}
                            child {node {\textbf{Agricultural robots}}}
                        }
                    }
                }
            ;
        \end{tikzpicture}

        \vspace{10px}
        \hrule
        \vspace{10px}

        One could divide the types of robotic applications into two main groups. First the group of \emph{industrial robots}, and second the group of \emph{service robots}. 
        Industrial robots are robots, which are used in the modern factories in highly automated scenarios. The \emph{International Standardization Organization} (ISO) defines them as "fixed in place or mobile for use in industrial automation application". 
        
        Service Robots on the other hand are defined by ISO as:
        \begin{displayquote}
            "Robot that performs useful tasks for humans or equipment excluding industrial automation applications."
        \end{displayquote} 

        Since this definition is still ambiguous, the ISO splits the field of service robots further into the two subfields of \emph{personal} and \emph{professional service robots}. \emph{Personal service robots} are robots used in non-commercial tasks by layman. Examples are robot vacuums, and general domestic servant robots. \emph{Professional service robots} in the other hand are the big field of robots used in commercial tasks by professionals. Examples are logistical, construction or military robots.
        Now in the field of professional service robots, exists the category of \emph{field robots}. Field robots are defined by the \emph{Robotic Industries Association} as:
        \begin{displayquote}
            "Field robots are non-factory mobile robots that operate in dynamic, unstructured environments. These types of robots aren’t programmed to repeat the same task over and over – they’re adaptive, responsive robots that work under variable conditions, sometimes even in unexplored territory. They often perform tasks that are too laborious or dangerous for humans. For this reason, as the underlying technology steadily advances their capabilities, field robots are becoming highly desirable to a number of different industries."
        \end{displayquote}

        Agricultural robots are a specific subfield of field robotics. Agricultural robots are deployed in agricultural industries and are mainly used for farming crops. For a better clarification of the positioning of agricultural robots in the field of robotics, refer to the figure \hyperlink{fig:category_mindmap}{\emph{Categorization of agricultural robots.}}

        \section{What are multi-robot systems, and what are their advantages?}

        Multi-robot systems (MRS) are teams of robots that perform together, trying to achieve a collective goal. When multiple robots work together, the collective can reach goals, which are impossible to reach for a single robot. Another benefit is, that MRS compared to single robots are more robust, more flexible and achieve higher performance. \cite{Albani2017}
        There are two main types of teams in MRS, namely homogeneous teams and heterogeneous teams. Homogenous teams are groups where each member is the same type of robot. These type of group is great to use, when the tasks to be done are easily distributed over the robots. Examples are teams of UAVs for scouting tasks, where each member of the team is responsible for a distinct part of the total area. 
        In heterogeneous teams, the members of the group can be of different robot types. This makes specialization between the robots possible. To come back to the UAV example, we could add UGVs to the group with some kind of manipulator for plant care. Now the UAVs can scout the area and deploy the UGV to positions where they are needed. 
        With the second example, it is possible to show some of the challenges associated with MRS: \ \begin{description}
            \item[Multi robot task allocation (MRTA)] The MRTA is the problem to be solved, for answering the question: Who is doing what?
            \item[Communication] When deploying multiple robots, it has to be defined, who is talking with whom? Are all data shared with everyone, or is this transferred to a communication hub first?
            \item[Self organization] How can the individual robot organize them selves?  
        \end{description}

        The MRTA problem is of central nature in the research of MRS. Finding an optimal solution for the question of who is doing what is a non trivial problem. Refer to \cite{Gerkey2004} for more in-depth explanation. There are two big organization approaches in MRS, first a centralized approach, where one central agent assigns the tasks to the robots, or a decentralized approach, in which the robots negotiate the task allocate themselves. A special case of the later approach is swarm robotics, here the robots only communicate with small subset of the whole group. This approach has very interesting characteristics, especially a great scalability, much higher robustness and flexibility \cite{Albani2017}.
        For a better distinguishing between different types of coordination in MRS, \cite{Farinelli2004} proposed a following taxonomy, which I recreated as mind map: 

        \vspace{10px}
        \hrule
        \vspace{10px}

        \begin{tikzpicture}[ align=flush center, level distance=1cm, sibling distance=3.5cm]
            \node[label=above:\textbf{Categorization cooperation in MRS}]{\hypertarget{fig:mrs_mindmap}{Cooperation}}
                child { node {Unaware}}
                child { node {Aware}
                    child {node {Not coordinated}}
                    child {node {Weakly coordinated}}
                    child {node {Strongly coordinated}
                        child {node {Not centralized}}
                        child {node {Weakly centralized}}
                        child {node {Strongly centralized}}
                    }
                }
            ;
        \end{tikzpicture}

        \vspace{10px}
        \hrule
        \vspace{10px}

        The reviewed work present in my reports are either strongly coordinated and strongly centralized or weakly coordinated and not centralized approaches. For the first category, the authors implemented hard oversight and mission planning \cite{deSantos2016}, the second approach is about swarms, which are only coordinating locally \cite{Albani2019}



    \chapter{Description of the reviewed papers}

    The biggest focus in this work lies at understanding the problems agricultural robots face, and second how MRS can help. In this section I will give a short overview over the topic I have found particularly interesting. 
    
    For convenient access to the papers presented here, refer to the table \hyperref[table:main_topics]{\emph{Main topics and categories of my papers}}. Because I needed to save space, the papers are only referred by a key composed of their main author`s name and the year published - but each key is hyper linked to the reading report for more information, including the full citation.

    My first goal was to get a overview over the topics of agricultural robots, so I searched for review papers, some older and some more current to get a feel for the progress and change in the field: 
    \begin{itemize}
        \item \hyperref[sec:Bac2014]{Reading report:} \cite{Bac2014}
        \item \hyperref[sec:Bechar2016]{Reading report:} \cite{Bechar2016}
        \item \hyperref[sec:Davidson2020]{Reading report:} \cite{Davidson2020}
        \item \hyperref[sec:Haidegger2013]{Reading report:} \cite{Haidegger2013}
        \item \hyperref[sec:Pedersen2006]{Reading report:} \cite{Pedersen2006}
        \item \hyperref[sec:Roldan2016]{Reading report:} \cite{Roldan2016}
        \item \hyperref[sec:Siciliano2016]{Reading report:} \cite{Siciliano2016}
        \item \hyperref[sec:Thorpe2001]{Reading report:} \cite{Thorpe2001}
        \item \hyperref[sec:Zhao2016]{Reading report:} \cite{Zhao2016}
    \end{itemize}
    After this, I changed my focus on harvesting robots, because harvesting is one of the most important and surely most fascinating topics of agricultural robotics: 
    \begin{itemize}
        \item \hyperref[sec:Arad2020]{Reading report:} \cite{Arad2020}
        \item \hyperref[sec:Henten2003]{Reading report:} \cite{Henten2003}
        \item \hyperref[sec:Herck2020]{Reading report:} \cite{Herck2020}
        \item \hyperref[sec:Lehnert2020]{Reading report:} \cite{Lehnert2020}
        \item \hyperref[sec:Lili2017]{Reading report:} \cite{Lili2017}
        \item \hyperref[sec:Wu2020]{Reading report:} \cite{Wu2020}
        \item \hyperref[sec:Weiss2011]{Reading report:} \cite{Weiss2011}
    \end{itemize}
    With the topics acquired so far, I took a turn into investigating MRS systems and looked for groundwork papers and reviews: 
    \begin{itemize}
        \item \hyperref[sec:Arai2002]{Reading Report:} \cite{Arai2002}
        \item \hyperref[sec:Brambilla2013]{Reading Report:} \cite{Brambilla2013}
        \item \hyperref[sec:Chamanbaz2017]{Reading Report:} \cite{Chamanbaz2017}
        \item \hyperref[sec:Gerkey2004]{Reading Report:} \cite{Gerkey2004}
        \item \hyperref[sec:Lerman2006]{Reading Report:} \cite{Lerman2006}
        \item \hyperref[sec:Khamis2015]{Reading Report:} \cite{Khamis2015}
        \item \hyperref[sec:Korsah2013]{Reading Report:} \cite{Korsah2013}
    \end{itemize}
    The next goal for me, was to find papers which connect the two topics: 
    \begin{itemize}
        \item \hyperref[sec:Albani2017]{Reading Report:} \cite{Albani2017}
        \item \hyperref[sec:Albani2019]{Reading Report:} \cite{Albani2019}
        \item \hyperref[sec:Bechar2017]{Reading Report:} \cite{Bechar2017}
        \item \hyperref[sec:ConesaMunoz2015]{Reading Report:} \cite{ConesaMunoz2015}
        \item \hyperref[sec:deSantos2016]{Reading Report:} \cite{deSantos2016}
        \item \hyperref[sec:deSantos2020]{Reading Report:} \cite{deSantos2020}
        \item \hyperref[sec:Vasconez2019]{Reading Report:} \cite{Vasconez2019}
    \end{itemize}

    All in all I am quite happy with the papers I have found and reviewed. They contain a great knowledge basis and also cutting edge research. I think my favorite paper so far was \emph{Development of a sweet pepper harvesting robot} \cite{Arad2020}. They not only had a good end result, but also described the software created detailed but also the newly, custom developed manipulator. I also liked their error analysis in particular. It was quite hard to compare between many of the papers, because every research group is using different methodologies, but if one of them is a great example on how to do it, then it is this one.
   

\begin{table}[]
    \centering
    \captionof{table}{Main topics and categories of my papers} \label{table:main_topics} 
    \rowcolors{1}{}{lightgray}
    \begin{tabular}{@{}lllllllllllllll@{}}
    \toprule
    Paper           & \rot{Review} & \rot{Agriculture} & \rot{MRS} & \rot{Swarm} & \rot{Field robotis} & \rot{Harvesting} & \rot{Scouting} & \rot{Weeding} & \rot{Communication} & \rot{Vision system} & \rot{Manipulation} & \rot{Simulation} & \rot{Lab experiment} & \rot{Field experiment} \\ \midrule
    \hyperref[sec:Albani2017]{Albani2017}      &        &  \checkmark          &  \checkmark  &  \checkmark    &  \checkmark            &            &  \checkmark       &  \checkmark      &  \checkmark            &  \checkmark            &              &  \checkmark         &                &                  \\
    \hyperref[sec:Albani2019]{Albani2019}      &        &  \checkmark          &  \checkmark  &  \checkmark    &  \checkmark            &            &  \checkmark       &  \checkmark      &  \checkmark            &  \checkmark            &              &  \checkmark         &  \checkmark             &                  \\
    \hyperref[sec:Arad2020]{Arad2020}        &        &  \checkmark          &     &       &  \checkmark            &  \checkmark         &          &         &               &  \checkmark            &  \checkmark           &            &                &  \checkmark               \\
    \hyperref[sec:Arai2002]{Arai2002}        &  \checkmark     &             &  \checkmark  &       &  \checkmark            &            &          &         &               &               &              &            &                &                  \\
    \hyperref[sec:Bac2014]{Bac2014}         &  \checkmark     &  \checkmark          &     &       &  \checkmark            &  \checkmark         &          &         &               &  \checkmark            &  \checkmark           &            &                &                  \\
    \hyperref[sec:Bechar2016]{Bechar2016}      &  \checkmark     &  \checkmark          &     &       &  \checkmark            &  \checkmark         &  \checkmark       &  \checkmark      &               &  \checkmark            &  \checkmark           &            &                &                  \\
    \hyperref[sec:Bechar2017]{Bechar2017}      &  \checkmark     &  \checkmark          &  \checkmark     &       &  \checkmark            &  \checkmark         &  \checkmark       &  \checkmark      &               &  \checkmark            &  \checkmark           &            &     \checkmark       &        \checkmark            \\
    \hyperref[sec:Brambilla2013]{Brambilla2013}   &  \checkmark  &             &  \checkmark  &  \checkmark    &               &            &          &         &  \checkmark            &               &              &            &                &                  \\
    \hyperref[sec:Chamanbaz2017]{Chamanbaz2017}   &        &             &  \checkmark  &  \checkmark    &  \checkmark            &            &          &         &  \checkmark            &  \checkmark            &              &  \checkmark         &  \checkmark             &  \checkmark               \\
    \hyperref[sec:ConesaMunoz2015]{ConesaMunoz2015} &        &  \checkmark          &  \checkmark  &       &  \checkmark            &            &          &  \checkmark      &  \checkmark            &  \checkmark            &              &            &  \checkmark             &  \checkmark               \\
    \hyperref[sec:Davidson2020]{Davidson2020}    &  \checkmark     &  \checkmark          &     &       &  \checkmark            &            &          &         &               &               &  \checkmark           &            &                &                  \\
    \hyperref[sec:deSantos2016]{deSantos2016}    &        &  \checkmark          &     &       &  \checkmark            &            &  \checkmark       &  \checkmark      &               &  \checkmark            &              &  \checkmark         &                &                  \\
    \hyperref[sec:deSantos2020]{deSantos2020}    &  \checkmark     &  \checkmark          &     &       &  \checkmark            &  \checkmark         &  \checkmark       &  \checkmark      &  \checkmark            &  \checkmark            &  \checkmark           &            &                &                  \\ 
    \hyperref[sec:Gerkey2004]{Gerkey2004}      &  \checkmark     &             &  \checkmark  &       &               &            &          &         &  \checkmark            &               &              &            &                &                  \\
    \hyperref[sec:Haidegger2013]{Haidegger2013}   &  \checkmark     &             &     &       &  \checkmark            &            &          &         &               &               &              &            &                &                  \\
    \hyperref[sec:Henten2003]{Henten2003}      &        &  \checkmark          &     &       &  \checkmark            &  \checkmark         &          &         &               &  \checkmark            &  \checkmark           &            &                &  \checkmark               \\
    \hyperref[sec:Herck2020]{Herck2020}      &        &  \checkmark          &     &       &  \checkmark            &  \checkmark         &          &         &               &  \checkmark            &  \checkmark           &            &                &  \checkmark               \\
    \hyperref[sec:Khamis2015]{Khamis2015}      &  \checkmark     &             &  \checkmark  &       &               &            &          &         &  \checkmark            &               &              &            &                &                  \\
    \hyperref[sec:Korsah2013]{Korsah2013}      &  \checkmark     &             &  \checkmark  &       &               &            &          &         &  \checkmark            &               &              &            &                &                  \\
    \hyperref[sec:Lehnert2020]{Lehnert2020}     &        &  \checkmark          &     &       &  \checkmark            &  \checkmark         &          &         &               &  \checkmark            &  \checkmark           &            &  \checkmark             &  \checkmark               \\
    \hyperref[sec:Lerman2006]{Lerman2006}      &        &             &  \checkmark  &       &               &            &          &         &  \checkmark            &               &              &            &                &                  \\
    \hyperref[sec:Lili2017]{Lili2017}        &        &  \checkmark          &     &       &  \checkmark            &  \checkmark         &          &         &               &  \checkmark            &              &            &                &  \checkmark               \\
    \hyperref[sec:Pedersen2006]{Pedersen2006}    &  \checkmark     &  \checkmark          &     &       &  \checkmark            &  \checkmark         &  \checkmark       &  \checkmark      &               &               &              &            &                &                  \\
    \hyperref[sec:Roldan2016]{Roldan2016}      &        &  \checkmark          &  \checkmark  &       &  \checkmark            &            &  \checkmark       &         &               &  \checkmark            &              &            &                &  \checkmark               \\
    \hyperref[sec:Siciliano2016]{Siciliano2016}   &  \checkmark     &  \checkmark          &     &       &  \checkmark            &  \checkmark         &  \checkmark       &  \checkmark      &               &  \checkmark            &  \checkmark           &            &                &  \checkmark               \\
    \hyperref[sec:Thorpe2001]{Thorpe2001}      &  \checkmark     &             &     &       &  \checkmark            &            &          &         &               &               &              &            &                &                  \\
    \hyperref[sec:Vasconez2019]{Vasconez2019}    &  \checkmark     &  \checkmark          &     &       &  \checkmark            &            &          &         &  \checkmark            &               &              &            &                &                  \\
    \hyperref[sec:Weiss2011]{Weiss2011}       &        &  \checkmark          &     &       &  \checkmark            &            &  \checkmark       &         &               &  \checkmark            &              &  \checkmark         &  \checkmark             &                  \\
    \hyperref[sec:Wu2020]{Wu2020}          &        &  \checkmark          &     &       &  \checkmark            &  \checkmark         &          &         &               &  \checkmark            &              &            &                &  \checkmark               \\
    \hyperref[sec:Zhao2016]{Zhao2016}        &  \checkmark     &  \checkmark          &     &       &  \checkmark            &  \checkmark         &          &         &               &  \checkmark            &              &            &                &                  \\ \bottomrule
    \end{tabular}
    \bigskip
    \centering

    \small Each papers key is hyper linked to the reading report for convenient access to more information.
    \end{table}

    \section{Methods used for finding research}

    In the beginning of my research, I mostly used \href{https://scholar.google.com}{Google Scholar} and looked up combinations of the my base keywords. The combination are presented in my \hyperref[sec:keywords]{appendix}. My base keywords were:
    \begin{itemize}
        \item Review
        \item Agriculture
        \item MRS
        \item Swarm
        \item Field robotics
        \item Harvesting
        \item Scouting
        \item Weeding
        \item Communication
        \item Vision system
        \item Manipulation
        \item Simulation
        \item Lab experiment
        \item Field experiment
    \end{itemize}

    After finding the first round of papers, I started to look trough journal checking the authors on citeseerX for credibility. My most used journals are \emph{Biosystems Engineering}, \emph{Computers and Electronics in Agriculture} and \emph{Precision Agriculture}.
     Another method I used, have been the reviewed papers referenced in review papers. This speed up my finding process immensely. Especially for research in the domain of harvesting robots was this essential, because the review authors already included important performance metrics.
        



\end{document}

