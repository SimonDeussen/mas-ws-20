%!TEX root = ../annotations.tex

\begin{document}
    \chapter{Appendix}

 

    \section{Acronyms}
    \begin{description}
        \item[ACML] Adaptive Monte Carlo localization
        \item[GNSS] Global navigation satellite systems
        \item[GPS] Global positioning satellites
        \item[HRI] Human-robot interaction
        \item[KPI] Key performance indicator
        \item[MRS] Multi-robot system
        \item[MRTA] Multi-robot task allocation
        \item[ROS] Robot operating system
        \item[RTK-GPS] Real time kinematic (with) global positioning satellite
        \item[SLAM] Simultaneous localization and mapping
        \item[UAV] Unmanned aerial vehicle
        \item[UGV] Unmanned ground vehicle
    \end{description}
    
    \section{Glossary}
    
    \begin{description}
        \item[Agricultural robotics] "An agricultural robot, is a robot designed for use in the agriculture industry. agricultural robots automate tasks for farmers, boosting the efficiency of production and reducing the industry’s reliance on manual labor. One of the biggest advantages of agricultural robots is that they can operate 24/7, 365 days a year. And, unlike human labor, they do not need to be paid – just maintained." \cite{MarketBusinessNews2020} 
        \item[Agriculture] "The science, art, or practice of cultivating the soil, producing crops, and raising livestock and in varying degrees the preparation and marketing of the resulting products" \cite{MerriamWebster2020}
        \item[Autonomous robots] "An autonomous robot is a robot that is designed and engineered to deal with its environment on its own, and work for extended periods of time without human intervention. Autonomous robots often have sophisticated features that can help them to understand their physical environment and automate parts of their maintenance and direction that used to be done by human hands." \cite{Technopedia2020}    
        \item[Crop engineering]  Modifying the crop and its environment to best fit a specific goal. In this context the crop gets modified in a way that allows autonomous robots to work with it. For example the crop should maximize the visibility and reachability of the fruits for easier harvesting.
        \item[Crop scouting]  "Crop scouting, also known as field scouting, is the very basic action of traveling through a crop field while making frequent stops for observations. Crop scouting is done so that a farmer can see how different areas of his or her field are growing. If there are problems during the growing season, the farmer can work to mitigate them so those problems do not affect yield at harvest time. Should problems go unnoticed or uncared for during the growing season, they can potentially limit the total yield, thus reducing the revenue from the sale of the crop or other intentions for the crop, such as livestock feed." \cite{Farms2020}
        \item[Distributed multi-robot system] "In a distributed MRS there is no centralized control mechanism—instead, each robot operates independently under local sensing and control, with coordinated system-level behavior arising from local inter- actions among the robots and between the robots and the task environment." \cite{Lerman2006}
        \item[Field robots] "Field Robots are machines that work in unstructured environments including under water, in mines, in forests and on farms, and in the air. [...] Field Robots is concerned with the automation of vehicles and platforms operating in harsh, unstructured environments. Field robotics encompasses the automation of many land, sea and air platforms in applications such as mining, cargo handling, agriculture, underwater exploration and exploitation, highways,planetary exploration, coastal surveillance and rescue, for example. Field robotics is characterized by the application of the most advanced robotics principles in sensing, control, and reasoning in unstructured and unforgiving environments. The appeal of field robotics is that it is challenging science, involves the latest engineering and systems design principles, and offers the real prospect of robotic principles making a substantial economic and social contribution to many different application areas" \cite{Thorpe2001}
        \item[Greenhouse agriculture] Opposed to open field agriculture, the growing is done in greenhouses with more possibilities to control the environment. It is possible to control the temperature, humidity, $CO_2$ concentration and the incoming solar radiation to find an optimum climate for the growing crops.
        \item[High value crop] " A major reason for a crop to be classified as a high-value crop is the high labor input required" \cite{Bac2014}
        \item[Human-robot interaction] "Human-robot interaction (HRI) is an emerging research field focused on the study of physical, cognitive and social interaction among people and robots, which can extend and improve human capabilities and skills. It is focused on designing, understanding, and evaluating the interaction between humans and robots that can communicate and/or share physical space" \cite{Vasconez2019}
        \item[Multi-robot system] "A system consisted of multiple robots which can cooperate and communicate with each other to accomplish certain tasks." \cite{IGIGlobal2020}
        \item[Multi-robot task allocation problem] "Task allocation in a multi-robot system is the problem of determining which robots should execute which tasks in order to achieve the overall system goals. Its purpose is coordinated team behavior. In some systems, such as some biologically inspired robotic systems, coordinated team behavior emerges as a result of local interactions between members of a team and with the environment. Its purpose is coordinated team behavior. In some systems, such as some biologically inspired robotic systems, coordinated team behavior emerges as a result of local interactions between members of a team and with the environment. This is referred to as implicit or emergent coordination. We are interested instead in explicit or intentional cooperation in which tasks are explicitly assigned to a robot or sub-team of robots, a problem described as multi-robot task allocation (MRTA)." \cite{Korsah2013}
        \item[Open field agriculture] Growing crops outside in fields, orchards or yards. Farming here is more season dependant than in greenhouses but harder to control and estimate.
        \item[Plant care] While farming often uses climate control to find the optimal environment, plant care tries to manipulate the plant for optimal growing condition. For example: removal of leaves, fruits or pests.
        \item[Plant mapping] This is the process of identifying and modelling every individual plant in a field or greenhouse. Plant mapping can be done using cameras of laser scanners. 
        \item[Precision agriculture] "Precision agriculture, also known as precision farming, is a broad term commonly used to describe particular farm management concepts, sometimes referred to as satellite farming or site specific crop management. The term first came into popular use with the introduction of GPS and GNSS as well as other methods of remote sensing which allowed farm operators to create precision maps of their fields that provide detailed information on their exact location while in-field. Advancements in technology have enabled the practice of precision agriculture to expand, providing even greater advantages for farmers and agricultural operations, including yield monitoring and crop scouting." \cite{Farms2020}
        \item[Robotic harvesting] Collecting and picking of crops done by robots.
        \item[Service robot] "A service robot is a robot which operates semi or fully autonomously to perform services useful to the well-being of humans and equipment, excluding manufacturing operations" \cite{InternationalFederationOfRobotics2021}
        \item[Task allocation] "Task allocation is a collective behavior in which robots distribute themselves over different tasks. The goal is to maximize the performance of the system by letting the robots dynamically choose which task to perform" \cite{Brambilla2013}
        \item[Utility] "Utility is a unifying, if sometimes implicit, concept in economics, game theory, and operations research, as well as in multi-robot coordination. It is based on the notion that each individual can internally estimate the value (or the cost) of executing an action. Depending on the context, utility is also called fitness, valuation, and cost." \cite{Gerkey2004}
        \item[Yield estimation] "One common desire of all fruit growers is knowledge of the crop yield. Accurate yield prediction helps growers improve fruit quality and reduce operating cost by making better decisions on intensity of fruit thinning and size of the harvest labor force. It benefits the packing industry as well, because managers can use estimation results to optimize packing and storage capacity. Typically, yield estimation is performed based on historical data, weather conditions, and workers manually counting fruit in multiple sampling locations" \cite{Siciliano2016}
    \end{description}

    \section{Sources}

    \subsection{List of searched journals}
    \begin{itemize}
        \item Agronomy
        \item Biosystems Engineering
        \item Computers and Electronics in Agriculture
        \item Computers {\&} Industrial Engineering
        \item IEEE Transactions on Robotics
        \item Image and Vision Computing
        \item International Journal of Advanced Robotic Systems
        \item Journal of Computational Science
        \item Journal of Control, Measurement, and System Integration
        \item Journal of Field Robotics
        \item Journal of Intelligent {\&} Robotic Systems
        \item Journal of Service Management
        \item Precision Agriculture
        \item Remote Sensing
        \item Robotics and Autonomous Systems
        \item Swarm and Evolutionary Computation
        \item Systems Science {\&} Control Engineering
        \item The International Journal of Robotics Research
        \item Transactions on Mechatronics
    \end{itemize}
    \subsection{List of searched conference proceedings}
    \begin{itemize}
        \item AAAI Conference on Artificial Intelligence 
        \item IEEE International Conference on Advanced Video and Signal Based Surveillance
        \item IEEE International Conference on Control Automation 
        \item IEEE International Conference on Field and Service Robotics
        \item IEEE International Conference on Robotics and Automation
        \item IEEE International Conference on Robotics, Intelligent Systems and Signal Processing
        \item IEEE International Conference on Real-time Computing and Robotics
        \item IEEE Systems Conference
        \item International Conference on Mechatronics and Automation
        \item Proceedings of the National Academy of Sciences
    \end{itemize}
    \subsection{List of searched magazine}
    \begin{itemize}
        \item Advances in Robotics {\&} Automation
    \end{itemize}
    \subsection{Other searched publications}
    \begin{itemize}
        \item American Society of Agricultural and Biological Engineers Annual International Meeting
        \item Experimental Robotics: The 13th International Symposium on Experimental Robotics
    \end{itemize}
    \subsection{Key words and key word combinations used for search}
    \begin{itemize}
        \item Agriculture: \ \begin{itemize}
            \item agricultural vision systems
            \item approach strategies harvesting robots
            \item autonomous agricultural robot
            \item design and implementation harvesting robot
            \item detection and control agricultural robot
            \item field experiments agricultural robotics
            \item high value crops harvesting robots
            \item multi-robot systems for agricultural uses
            \item performance evaluation harvesting robot        
            \item review state-of-the-art agricultural robotics
            \item robotic harvester performance evaluation
            \item robotic harvesting system
            \item sequence of tasks harvesting robots
            \item swarm in agricultural robotics
            \item swarm in agricultural robotics performance
        \end{itemize}
        \item MRS: \ \begin{itemize}
            \item motion control of multi robot
            \item multi-robot system  communication
            \item multi-robot system cooperative exploration
            \item multi-robot systems control framework
            \item multi-robot systems hierarchical controller
            \item multi-robot systems local path planner
            \item swarm robotics
            \item swarm robotics advantages
            \item swarm robotics review survey
        \end{itemize}
        
    \end{itemize}
    \subsection{List of most important conferences}
    \begin{itemize}
        \item International Conference on Field and Service Robotics
        \item International Conference on Robotics, Intelligent Systems and Signal Processing
        \item IEEE International Conference on Robotics and Automation
    \end{itemize}
    \subsection{List of most important journals and magazines}
    \begin{itemize}
        \item Computers {\&} Industrial Engineering
        \item Biosystems engineering
        \item Journal of Field Robotics
    \end{itemize}
    \subsection{List of top research labs/researchers}
    \begin{itemize}
        \item A. Bechar 
        \item B. Gerkey
        \item C. Thorpe 
    \end{itemize}
    \subsection{Link collection of online literature search}
    \begin{itemize}
        \item http://precisionagricultu.re/
        \item https://ifr.org/service-robots/
        \item https://marketbusinessnews.com/financial-glossary/agricultural-robot-agribot/
        \item https://www.elsevier.com/
        \item https://www.fao.org
        \item https://www.farms.com/
        \item https://www.researchgate.net/
        \item https://www.robotics.org/service-robots/what-are-professional-service-robots
        \item https://www.sciencedirect.com/
    \end{itemize}
    \subsection{Figures}

    \subsubsection{Mindmaps}
    \begin{itemize}
        \item \hyperlink{fig:mindmap_agribots}{Agricultural robots}
        \item \hyperlink{fig:category_mindmap}{Categorization of agricultural robots}
        \item \hyperlink{fig:mindmap_agribots}{Agricultural robots}
    \end{itemize}
    \subsubsection{Tables}
    \begin{itemize}
        \item \hyperref[table:objects_and_tasks]{Comparison of objects and tasks}
        \item \hyperref[table:paper_types]{Paper types}
        \item \hyperref[table:main_topics]{Main topics}
    \end{itemize}
\end{document}
