\documentclass{article}

\usepackage{graphicx}
\usepackage{hyperref}
\usepackage{apacite}

\title{MAS ISW Reading Reports}
\date{05.12.2020}
\author{Simon Deussen}

\begin{document}
\pagenumbering{gobble}
\maketitle
\pagenumbering{arabic}

% ================= TEMPLATE =================
% \section{Reading Report: \emph{TITLE}}
% \cite{}

% \subsection*{Abstract}

% \subsection*{Keywords}

% \subsection*{Questions}

% \subsubsection*{What are the motivations for this work?}
% \subsubsection*{What is the proposed solution?}
% \subsubsection*{What is the work`s evaluation of the solution?}
% \subsubsection*{What is my analysis of the identified problem, idea  and evaluation?}
% \subsubsection*{What are the contributions?}
% \subsubsection*{What are the future directions of the research?}
% \subsubsection*{What questions have I left?}
% \subsubsection*{What is my main take away from this paper?}

% \subsection*{Summary}

% \subsection*{Rating}

% ================= TEMPLATE =================

\section{Reading Report: \emph{Agricultural robots for field operations: Concepts and components}}
\cite{Bechar2016}

\subsection*{Abstract}

This review investigates the research effort, developments and innovation in agricultural
robots for field operations, and the associated concepts, principles, limitations and gaps.
Robots are highly complex, consisting of different sub-systems that need to be integrated
and correctly synchronised to perform tasks perfectly as a whole and successfully transfer
the required information. Extensive research has been conducted on the application of
robots and automation to a variety of field operations, and technical feasibility has been
widely demonstrated. Agricultural robots for field operations must be able to operate in
unstructured agricultural environments with the same quality of work achieved by cur-
rent methods and means. To assimilate robotic systems, technologies must be developed
to overcome continuously changing conditions and variability in produce and environ-
ments. Intelligent systems are needed for successful task performance in such environ-
ments. The robotic system must be cost-effective, while being inherently safe and
reliabledhuman safety, and preservation of the environment, the crop and the machinery
are mandatory. Despite much progress in recent years, in most cases the technology is not
yet commercially available. Information-acquisition systems, including sensors, fusion
algorithms and data analysis, need to be adjusted to the dynamic conditions of un-
structured agricultural environments. Intensive research is needed on integrating human
operators into the system control loop for increased system performance and reliability.
System sizes should be reduced while improving the integration of all parts and com-
ponents. For robots to perform in agricultural environments and execute agricultural
tasks, research must focus on: fusing complementary sensors for adequate localisation
and sensing abilities, developing simple manipulators for each agricultural task, devel-
oping path planning, navigation and guidance algorithms suited to environments besides
open fields and known a-priori, and integrating human operators in this complex and
highly dynamic situation.


\subsection*{Keywords}
Agricultural robots, Robotics, Field operations, Autonomous


\subsection*{Questions}

\subsubsection*{What are the motivations for this work?}
The main subject of this paper is to show the current development, ideas and problems in the field of agricultural 
robotics. This review paper explains first the background, then the economic feasibility and furthers goes into 
concepts, principles and components.
\subsubsection*{What is the proposed solution?} 
The paper concludes, that with current technologies the broad usage in commercial farming is not possible yet and 
proposes to focus research on a number of fields. Those fields include sensor fusion for better localisation, 
engineering of better simple manipulators and the development of specific path planning, navigation and guidance algorithms
for agriculture.
\subsubsection*{What is the work`s evaluation of the solution?}
This question is not applicable.
\subsubsection*{What is my analysis of the identified problem, idea  and evaluation?}
The authors make a great job in displaying the current technologies and their limitations. With this knowledge it is 
easy to identify a subproblem to work on.
\subsubsection*{What are the contributions?}
Several points come to the mind. Firstly they create an in-depth background needed to understand the need of automated
systems in agriculture, but also explain why it is so hard to create such systems. 
They propose a categorization of robotic system after the structure of their environment and object of interest. Both 
can be either structured or unstructured. This categorization creates four different categories. First, a structured 
environment and a structured object: This is the industrial domain. Second, a strucutred environment and a unstructured
object: the medial domain. Further there is the unstructured enviroment with a structured object: the military, space, underwater
andmining domains. The last domain, unstructured in environment and object of interest is the agricultural domain.

The next contribution are guidelines under which circumstances a robot can be commercially successful. These guideline conclude
that it is possible to start using robots even if the costs are the same as conventional methods if the work of
the robots create more steady and predictable processes.

A big part of the review are categorization concepts, components and principles. These include Human-Robot-Systems versus
Autonomous Robot Systems. In the component section the authors underline following topics: steering and mobility, 
sensing and self-localization, path planning and guidance and last but not least, manipulators and effectors.
\subsubsection*{What are the future directions of the research?}
This question is not applicable.
\subsubsection*{What questions have I left?}
Many questions, this paper is an excellent basis for further research.
\subsubsection*{What is my main take away from this paper?}
One of the main problems is the highly dynamic environment and the need to react fast to unprecedented situations.
This creates the question on how to define behavior in such a way to allow and strengthen the capabilities of 
improvisation.
\subsection*{Summary}
In-depth review paper with some self citations but besides that it gives many new points to deepen my reseach.

\subsection*{Rating}
5/5

\section{Reading Report: \emph{Agricultural robots—system analysis and economic feasibility}}
\cite{Pedersen2006}

\subsection*{Abstract}
This  paper focuses  on  the  economic  feasibility of  applying  autonomous robotic  vehicles  compared  to  conventional  systems  in  three  different  applications: robotic weeding in high value crops (particularly sugar beet), crop scouting in cereals and grass cutting on golf courses. The comparison was based on a systems analysis and an individual economic feasibility study for each of the three applications. The results  showed  that  in  all  three  scenarios,  the  robotic  applications  are  more  economically feasible than the conventional systems. The high cost of real time kinematics Global Positioning System (RTK-GPS) and the small capacity of the vehicles are the main parameters that increase the cost of the robotic systems.

\subsection*{Keywords}
Agricultural robots, Grass cutter, Autonomous vehicles, Economics, Feasibility study, Robotic weeding, Crop scouting

\subsection*{Questions}

\subsubsection*{What are the motivations for this work?}
The papers main focus lies on displaying the cost reduction possible by utilizing 
autonomous system for agriculture tasks. Most agricultural task can not use individual-plant-based solutions with
conventional methods. By using robots and big data processing it will be possible to care for each plant individually.
Taking care of an identified weed patch for example will need much less herbicides than spaying the whole field preemptively.
\subsubsection*{What is the proposed solution?}
The authors propose solutions for using autonomous robots for field scouting - the identification and localisation 
of growing weeds -, intra-row and near-crop weeding and automated grass cutting. 
\subsubsection*{What is the work`s evaluation of the solution?}
In all scenarios the authors showcase a reduction in primary and secondary costs in comparison to conventional methods.
\begin{description}
    \item[Field scouting] 20\% cost reduction in labor and secondary benefits of the data because it is now possible to only deploy herbicides where needed.
    \item[Weeding] Only by reducing the cost of the navigation system by half it is possible to save 12-21\% or manuel costs. and reduction of herbicide use of 90\%   
    \item[Grass cutting] Reduction of cost of 52\% (but only when paying the gardener 27 Euro per hour, lol) 
\end{description}
\subsubsection*{What is my analysis of the identified problem, idea  and evaluation?}
The usage of automated systems for growing crops is one of the key points in reducing the environmental footprint
of large scale agriculture. The three analyzed areas are great entry points for deploying such systems. Especially 
the field scouting and the automated weeding are very interesting. For the evaluation the authors compared 
the costs of the components with average conventional costs witch is mostly reasonable, expect the estimated
labor cost of the gardener of 27 Euros per hour for grass cutting.
\subsubsection*{What are the contributions?}
The ideas of the authors in breaking down the cost of the robots into several components are very helpful to estimate economic costs of different system for
this usage. The main contribution is this economic analysis which helped to spark more research in this direction.
\subsubsection*{What are the future directions of the research?}
There will always be economic analyses for newer technology.  
\subsubsection*{What questions have I left?}
Because this paper is from 2006 I am eager to find a similar, more current breakdown.
\subsubsection*{What is my main take away from this paper?}
That it is feasible to automate many agricultural tasks with almost existing technology.

\subsection*{Summary}
Great in depth analysis but dated (2006), has good numbers for conventional cost estimates.
\subsection*{Rating}
3/5

\bibliography{isw_research}
\bibliographystyle{apacite}

\end{document}