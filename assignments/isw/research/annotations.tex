\documentclass{report}
\usepackage{graphicx}
\usepackage{hyperref}
\usepackage{apacite}
\usepackage{geometry}

\usepackage{graphicx} % Required to include images
\usepackage{amsmath,amssymb,theorem} % Math packages
\usepackage{listings} % Required for including snippets of code
\usepackage{booktabs} % Required for better horizontal rules in tables
\usepackage{microtype} % Slightly tweak font spacing for aesthetics
\usepackage[titletoc]{appendix}
\usepackage{setspace}
\usepackage{subfiles}
% \usepackage{showframe}
\usepackage{subcaption}	% Required for sub-figures
\usepackage{multirow} % Required to enable merging of columns/rows in a table
\usepackage{rotating} % Required to display sideways tables/figures
\usepackage{hyperref} % Enable hyperlinks
\usepackage{titling}
\usepackage{csquotes}
\usepackage{lipsum} % Used to test
\usepackage{lmodern}%
\usepackage{sdtitle}

\usepackage{metalogo}
\usepackage[table]{xcolor}

\usepackage{tikz}
\usetikzlibrary{mindmap}

\hypersetup{%
colorlinks=true,  % Use colored links instead of boxes
linkcolor=black,  % Internal link color
citecolor=black,  % Citations color
urlcolor=black,  % External url color
pagebackref=true, % Enable reference links
pdftitle={MAS ISW Seminar Report},
pdfauthor={Simon Deussen}
}%


\title{Multi-robot system for agricultural uses}
\date{February 2021}
\author{Simon Deussen}

\begin{document}

\begin{titlepage}
    \maketitle
\end{titlepage}


\pagenumbering{arabic}

\newpage
\chapter*{Abstract} 

This work reports upon the state-of-the-art of agricultural robots, why they still fail to replace human labourers and if multi-robot approaches are a solution. 
The progress of automation in agricultural sector lacks far behind the industrial sector, although researchers have been trying for decades to introduce the same level of automation into the farms. In contrast to robots working in the highly engineered and controlled environment of industrial assembly lines, robots in agriculture have to deal with unstructured, dynamic and stochastic environments and objects. So far research has not solved the problems of modelling the environment in a way, that the agricultural robots are able to understand and act in them effectively. Thus the performance and production rate of current systems is too bad for economic justification. 

For this work the best 100 papers, from the areas of agricultural robots, multi-robot systems and everything in between, got researched and the top 30 of them got reported upon. In chapter 4 are detailed reading reports and thoughts about the their work written down.

After this report on the state-of-the-art, the recognized key findings are as follows: Though progress exists, and agricultural robots are getting better, the state-of-the-art is not ready for commercialization. Promising projects are using heterogenous or homogenous multi-robot approaches. Utilizing the strength of different types of robots or the intelligence of a collective are ways to step out of the limitations the technologies still has.

Concluding, for the last 30 years, researchers always thought that they are almost ready for commercialization in the field of agricultural robots. So far this has not happened, but novel approaches will be ready for a couple of use cases. The first multi-robot systems, monitoring crop health in vast fields in autonomous fashion should not be too far in the future.

Automation in agricultural domains is needed, and while there are many problems to tackle, agricultural robots are now closer then ever.


\tableofcontents
\newpage

\subfile{chapters/chpt_01}
\subfile{chapters/chpt_02}
\subfile{chapters/chpt_03}
\subfile{chapters/chpt_04}



\newpage
\bibliography{isw_research}
\bibliographystyle{apacite}

\end{document}
