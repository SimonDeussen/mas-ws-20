\documentclass{article}

\usepackage{graphicx}
\usepackage{hyperref}
\usepackage{apacite}

\title{MAS ISW Glossary for \emph{MRS in agriculture}}
\date{15.12.2020}
\author{Simon Deussen}

\begin{document}
\pagenumbering{gobble}
\maketitle
\pagenumbering{arabic}


\begin{description}
    \item[Agriculture] "The science, art, or practice of cultivating the soil, producing crops, and raising livestock and in varying degrees the preparation and marketing of the resulting products" \cite{MerriamWebster2020}
    \item[Agricultural robotics] "An agricultural robot, also known as an agribot, is a robot designed for use in the agriculture industry.
    Agribots automate tasks for farmers, boosting the efficiency of production and reducing the industry’s reliance on manual labor. 
    One of the biggest advantages of agribots is that they can operate 24/7, 365 days a year. And, unlike human labor, they do not need to be paid – just maintained." \cite{MarketBusinessNews2020} 
    \item[Autonomous robots] "An autonomous robot is a robot that is designed and engineered to deal with its environment on its own, and work for extended periods of time without human intervention. 
    Autonomous robots often have sophisticated features that can help them to understand their physical environment and automate parts of their maintenance and direction that used to be done by human hands." \cite{Technopedia2020}    
    \item[Crop engineering]  Modifying the crop and its environment to best fit a specific goal. In this context the crop gets modified in a way that allows autonomous robots to work with it. For example the crop should maximize the visibility and reachability of the fruits for easier harvesting.
    \item[Crop scouting]  "Crop scouting, also known as field scouting, is the very basic action of traveling through a crop field while making frequent stops for observations. Crop scouting is done so that a farmer can see how different areas of his or her field are growing. If there are problems during the growing season, the farmer can work to mitigate them so those problems do not affect yield at harvest time. Should problems go unnoticed or uncared for during the growing season, they can potentially limit the total yield, thus reducing the revenue from the sale of the crop or other intentions for the crop, such as livestock feed." \cite{Farms2020}
    \item[Robotic harvesting] Collecting and picking of crops done by robots.
    \item[Greenhouse agriculture] Opposed to open field agriculture, the growing is done in greenhouses with more possibilities to control the environment. It is possible to control the temperature, humidity, $CO_2$ concentration and the incoming solar radiation to find an optimum climate for the growing crops.
    \item[Open field agriculture] Growing crops outide in fields, orchards or yards. Farming here is more season dependant than in greenhouses but harder to control and estimate.
    \item[Plant care] While farming often uses climate control to find the optimal environment, plant care tries to manipulate the plant for optimal growing condition. For example: removal of leaves, fruits or pests.
    \item[Plant mapping] This is the process of identifying and modelling every individual plant in a field or greenhouse. Plant mapping can be done using cameras of laser scanners. 
    \item[Precision Agriculture] "Precision agriculture, also known as precision farming, is a broad term commonly used to describe particular farm management concepts, sometimes referred to as satellite farming or site specific crop management (SSCM). The term first came into popular use with the introduction of GPS (global positioning satellites) and GNSS (global navigation satellite systems) as well as other methods of remote sensing which allowed farm operators to create precision maps of their fields that provide detailed information on their exact location while in-field. Advancements in technology have enabled the practice of precision agriculture to expand, providing even greater advantages for farmers and agricultural operations, including yield monitoring and crop scouting." \cite{Farms2020}
    \item[Yield estimation] "One common desire of all fruit growers is knowledge of the crop yield. Accurate yield prediction helps growers improve fruit quality and reduce operating cost by making better decisions on intensity of fruit thinning and size of the harvest labor force. It benefits the packing industry as well, because managers can use estimation results to optimize packing and storage capacity. Typically, yield estimation is performed based on historical data, weather conditions, and workers manually counting fruit in multiple sampling locations" \cite{Siciliano2016}
    \item[Multi-robot Task Allocation (MRTA) Problem] "Task allocation in a multi-robot system is the problem
    of determining which robots should execute which tasks
    in order to achieve the overall system goals. Its purpose
    is coordinated team behaviour. In some systems, such as
    some biologically inspired robotic systems, coordinated
    team behavior emerges as a result of local interactions
    between members of a team and with the environment. Its purpose
    is coordinated team behaviour. In some systems, such as
    some biologically inspired robotic systems, coordinated
    team behavior emerges as a result of local interactions
    between members of a team and with the environment.
    This is referred to as implicit or emergent (Gerkey, 2003)
    coordination. We are interested instead in explicit or inten-
    tional (Parker, 1998) cooperation in which tasks are explic-
    itly assigned to a robot or sub-team of robots, a problem
    described as multi-robot task allocation (MRTA).
    "
     Korsah2013
    \item[Utility] Gerkey2004
    \item[Market-Based Approaches] khamis2015
    \item[Optimization-Based Approaches] khamis2015 
\end{description}

\newpage
\bibliography{glossary}
\bibliographystyle{apacite}

\end{document}